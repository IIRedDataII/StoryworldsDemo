\documentclass[12pt]{article}
%\usepackage{fontspec}
%\setmainfont{Calibri}
\usepackage{setspace}
\onehalfspacing
\usepackage{lineno}
\usepackage{geometry}
\usepackage[german]{babel}
\geometry{left=3cm, right=2.5cm, top=2.5cm, bottom=2cm}
\usepackage{hyperref}

\renewcommand\linenumberfont{\scriptsize}
\modulolinenumbers[5]
\title{Geschichtswelten Exposé}
\author{Timon Haas, Julian Zimmermann, Anian Kalb}
\date{}
\begin{document}
	\maketitle
	\tableofcontents
	\newpage
	\section{Konzept}
	\linenumbers
		Zu Beginn des Exposés wollen wir kurz das Konzept, welches hinter unserer Idee steht, erläutern.
	\subsection{Erfahrbarkeit Storyworld}
		Wir haben uns für eine Storyworld der Science-Fiction entschieden. Näheres zur Storyworld wird in Abschnitt 2 erläutert.\newline
		Die Storyworld soll in unserer Spieldemo durch verschiedene Methoden erlebt werden. Dabei möchten wir den Informationsfluss nicht immer direkt über Texte gestalten, sondern der Spieler soll durch die Interaktion des Hauptcharakters mit verschiedenen Objekten erfahren, welche Position dieser gegenüber ihnen einnimmt. Dadurch wird dem Spieler nach und nach die Storyworld erzählt und seine Situation verdeutlicht. Hierfür haben wir uns verschiedene Mittel überlegt, die im Folgenden aufgelistet sind:
		\begin{itemize}
			\item Erinnerungen der Hauptfigur per Rückblicksequenz
			\item Gespräch mit anderen Crewmitgliedern
			\item Datapads mit Forschungsergebnissen der Aliens über die Menschen
			\item Familienfotos und Besitztümer der Forschercrew
			\item Eingegangene Nachrichten im Computer des Forscherschiffs
		\end{itemize}
		Hinzu kommen auch visuelle Mittel, insbesondere die Gestaltung der Umgebung, der Architektur und des Planeten sollen dem Spieler mitteilen, dass sich dieser nicht auf der Erde befindet. Den Planeten selbst sieht der Spieler zum ersten mal über ein Aussichtsfenster, wodurch er die Stadt der Aliens und auch sein Raumschiff erkennt, welches zum ersten Zielort der Hauptfigur wird. Ein weiteres Indiz dafür, dass der Spieler sich auf einem fremden Planeten befindet, ist eine Begegnung mit der Alienrasse selbst.
	\subsection{Spielgenre}
		Für das Spiel haben wir uns für die vier folgenden Genres entschieden: Survival, Horror/Mystery, Puzzlesolving und Interactive Storytelling. Survival, da der Grundkonflikt unserer Storyworld das Überleben an sich ist und auch innerhalb des Spiel die Aktionen des Spielers ans Überleben gebunden sind. Horror, da sich der Spieler alleine in einem Labor auf einer fremden, feindseligen Welt wiederfindet. Mystery geht Hand in Hand mit der Idee des Horrors da die Unwissenheit des Spielers ein weiterer Grund ist, welcher Angst im Spieler hervorruft, aber auch für Spannung und Interesse an der Welt sorgt. Puzzlesolving macht einen relevanten Teil des Spiels aus, da dieses anfangs daraus besteht, dass der Spieler aus einem Labor entkommen muss. Mit Interactive Storytelling ist gemeint, dass der Ausgang des Spiels von den Entscheidungen des Spielers abhängt. Man kann durchaus eine \glqq falsche\grqq\ Entscheidung treffen, die in eine Sackgasse führt. Dadurch wird der Survival- sowie Horroraspekt unterstrichen.\\
		In der Spieldemo konnte der Horror des Spiels nur begrenzt umgesetzt werden.
	\subsection{Unique Selling Point}
		Das Spiel bietet mehrere Unique Selling Points. Zum einen bleibt die Welt in ihrem Design sehr realitätsnah was zum Beispiel interplanetares Reisen  oder das Konzept und die Folgen des Klimawandels betrifft.\newline 
		Ein weiterer Selling Point ist die interaktiv erlebbare Geschichte, welche sich durch das Verhalten des Spielers verändert, sodass es von diesem abhängt, wie viel er über die Welt erfährt, was den Ausgang des Spiels beeinflusst.\newline
		Der letzte Unique Selling Point das Spiel mit der Hoffnung. Der Hoffnung der Charaktere, sowie der des Spielers. Auf das Motiv der Hoffnung wird im Punkt \textbf{\textit{\hyperref[All.]{\ref{All.} \nameref{All.}}}} näher eingegangen.
		
	\newpage
	\section{Storyworld}
	
	
	\subsection{Überblick}
	Unsere Geschichte spielt in einer Welt, die sich grundsätzlich verhältnismäßig wenig von der realen Welt, wie wir sie kennen, unterscheidet. Es gibt hauptsächlich zwei Unterscheidungsmerkmale:
	
	\subsubsection{Futuristic Science Fiction}
	Wir bedienen uns dem klassischen Zukunftselement und dem damit verbundenen technologischen Fortschritt wie Raumschiffen und Kälteschlaf.\\
	Jedoch sei hier zu erwähnen, dass wir uns nah an der Realität halten wollen.\\\\
	Die meisten Sci-Fi Geschichten umgehen das Hauptproblem im Bezug auf Geschichten, deren Welten sich über die
	ganze Galaxie erstrecken, nämlich  das Problem der gigantischen Entfernungen, indem sie eine Möglichkeit
	etablieren, sich schneller als Lichtgeschwindigkeit fortzubewegen, siehe zum Beispiel den Hyperantrieb in Star Wars.\\
	Wir wollen davon ablassen. In unserer Welt werden große Entfernungen auf eine realistischere Art und Weise zurückgelegt: Die Raumschiffe haben einen Antrieb, der sie mit halber Lichtgeschwindigkeit reisen lässt. Die dabei verstrichene Zeit wird in der Regel im Kälteschlaf verbracht, während dem man nicht altert und der Stoffwechsel vorübergehend angehalten wird.\\
	\\
	Das sorgt in erster Linie dafür, dass der allgemeine Weltraum in unserer Welt kein offenes, zugängliches Gebiet ist, sondern ein unbekannter, riesiger Raum im Dunkeln bleibt.\\
	Ein auf einem fremden Planet gestrandeter Mensch kann nicht einfach zum nächsten Sonnensystem fliegen, da er von dort einen Guten-Morgen-Gruß per Überlichtgeschwindigkeitsfunk erhalten hat.\\
	Er muss sich genau überlegen, welches Ziel er als nächstes anpeilt, da sich die Welt, die ihn dort erwartet, in seiner Reisezeit möglicherweise über Jahrhunderte stark verändert hat.\\
	\\
	Vielleicht würden wir in Nachfolgespielen Singularitäten im Raum, wie Wurmlöcher, teilweise als alternative Lösung dieses Problems verwenden, aber nie so, dass man sie innerhalb der Welt geplant einsetzen kann, sondern nur als Zufall, der nun zwei Welten zeitweise miteinander verbindet.
	
	\subsubsection{Was wäre, wenn...?}
	Mit diesen Hintergrundregeln entsteht unser Setting nun mit den folgenden Fragen:\\
	Was wäre, wenn der Klimawandel bereits so weit fortgeschritten wäre, dass es sich nicht mehr verhindern ließe,
	dass die Erde in absehbarer Zeit für Menschen unbewohnbar wird?\\
	Was wäre, wenn des Weiteren die Menschheit Beweise dafür gefunden hätte, dass die Venus vor tausenden von
	Jahren noch bewohnbar war und eine intelligente Ur-Alienspezies beherbergt hatte, die jedoch aufgrund der
	Erwärmung der Venus, die dasselbe Schicksal ereilt hatte was nun der Erde bevorsteht, von ihr fliehen musste?\\
	\\
	Tatsächlich ist es in unserer Welt nun so, dass diese Ur-Aliens der Venus ihrer Zeit Raumschiffe in alle Richtungen entsandt hatte, um neuen Lebensraum zu erschließen. Sie sendeten ihre Bevölkerung im Kälteschlaf zu allen umliegenden Sonnensystemen, auf denen theoretisch Leben möglich wäre, ließen sich dort nieder und entwickelten sich in all dieser Zeit dort abgeschieden voneinander durch Evolution in unterschiedliche Spezies.\\
	Unser Sonnensystem ist nun also \glqq umringt\grqq\ von Leben, hervorgegangen von den ursprünglichen Venusbewohnern.\\
	\\
	Die Menschheit weiß all das zwar nicht genau, aber vermutet es und sendet nun ihrerseits diesem Beispiel
	folgend Raumschiffe los, um alternativen Lebensraum zu finden und hoffentlich die Erdbevölkerung umzusiedeln,
	bevor es zu spät ist.\\
	Nur ist ihre Situation insofern anders, als dass sie sich nicht leeren Planeten, sondern allerhand bevölkerten
	Planeten mit intelligenten Alienarten gegenübersehen.\\
	\\
	Das leitet nun auch die in unserer Story agierenden Gruppen und den Grundkonflikt ein.
	
	
	\subsection{Konflikt}
	Das Stichwort unseres allgemeinen Grundkonflikts lautet: Überleben.\\
	\\
	Überleben des Einzelnen (darauf gehen wir in unserem Spiel und der Main-Story näher ein), Überleben der Spezies durch Erobern des Lebensraumes und Erhalten einer Regierungsform, die das Überleben der Spezies sichert und auch die Art und Weise des Überlebens beziehungsweise des Lebens an sich spielt eine Rolle bei den Konflikten der Welt.
	
	
	\subsection{Gruppierungen}
	Generell ist jeder Gruppierung in unserer Welt bewusst, dass es in erster Linie um wertvollen Lebensraum
	geht, was den Hauptkonflikt dieser und auch zumindest zu einem großen Teil weiterer Stories innerhalb der Geschichtswelt ausmacht. Daher wird auch in anderen Geschichten immer eine Gruppe die Menschen und eine andere die Aliens sein, die sich dann weiter in Untergruppen aufspalten.\\
	\\
	Jedoch variieren die genaueren Gruppierungen von Story zu Story in der Storyworld immer etwas, weshalb ich die Gruppierungen in unserem Fall anhand des Beispiels unserer Story im Spiel erklären möchte.\\
	\\
	Im Spiel erzählen wir die Geschichte eines der Raumschiffe, das ausgesandt wurde, um Lebensraum zu finden. Dieses trifft auch auf einen lebensfreundlichen Planeten, der jedoch bereits von den sogenannten Chia bevölkert ist. Diese von den Ur-Aliens abstammende Lebensform hat eine menschenähnliche Grundform, die aber durch Merkmale wie Fühler, Rückenkamm oder der Fähigkeit, sich zum schnelleren Laufen auf alle viere fallen zu lassen, von Menschen unterscheidet, ist intelligent und hat ein sehr autoritäres, durch Religion begründetes Staatssystem, das sich vollkommen auf das Überleben der Spezies fokussiert und das Individuum komplett in den Hintergrund stellt.
	\subsubsection{Einschub: Religionserläuterung}
		Die Chia stammen wie bereits erwähnt von der Venus ab. Als Resultat des erlittenen Traumas, ihren alten Heimatplaneten zerstört zu haben, entwickelte sich auf dem neuen Planeten eine Religion, dessen oberstes Ziel es ist das Überleben der Spezies zu sichern. Daraus entstand die Legende der Erlösung bis zu der die Spezies überleben muss. Die Menschen werden als Bedrohung hierfür angesehen, da diese eine zerstörerische Lebensweise haben. Weiteres hierzu in \textbf{\textit{\hyperref[derivat]{\ref{derivat}. \nameref{derivat}}}}.
	\\Es entwickeln sich also vier Gruppierungen, von denen im Spiel eigentlich nur drei für den Verlauf relevant sind:
	\begin{itemize}
		\item Die Chia: Regierung
		\item Die Chia: Rebellen genannt Pandora
		\item Die Menschen: Raumschiffscrew
		\item Die Menschen: Erdbevölkerung (zu weit entfernt, um wirklich zu agieren)
	\end{itemize}\mbox{}\\
	Hier ist die Regierung der Chia daher den Menschen gegenüber feindlich gesonnen, da sie sie für eine potenzielle Gefahr für das Überleben ihrer Art halten.\\
	Jedoch spielt der Konflikt zwischen der Regierung und den Rebellen, die den Menschen helfen wollen, ebenfalls eine Rolle und erfüllt den oben erwähnten Punkt der Frage nach der Art und Weise des (Über)Lebens.\\
	\\
	Denn durch das strikte, autoritäre Staatssystem, das den Fokus vollkommen auf die Spezies als ganzes gelegt hat, mit allen Mitteln das Überleben dieser auf Dauer sichern will und dabei die Wünsche des einzelnen Individuums vollkommen außer Acht lässt, ist diese Rebellengruppe entstanden, deren Philosophie eine Gegenbewegung dazu darstellt. Für sie ist das Überleben der Spezies nichts wert, wenn die einzelnen Individuen kein lebenswertes Leben führen können.\\
	Aufgrund dieser Philosophie helfen sie dem letzten Überlebenden der auf ihrem Planeten gelandeten fremden Menschen. Sie erhoffen sich durch die Hilfe des Menschen das religiöse Staatssystem zu stürzten.\\
	\\
	Das Ziel der Raumschiffscrew ist es ursprünglich, den Menschen auf der Erde zu berichten, ob es hier Lebensraum gibt, um eine mögliche Umsiedlung in Gang zu setzen. Im Verlauf des Spiels verändert diese sich jedoch, da herauskommt, dass die Erdbevölkerung nicht mehr existiert. Mehr dazu bei \textbf{\textit{\hyperref[story]{\ref{story} \nameref{story}}}}.\\
	\\
	Wie diese Erdbevölkerung damit umgeht, dass eine potenzieller Planet existiert, der aber bereits bevölkert ist,
	hängt von der Fraktion der Erde ab, von der das Forschungsschiff ausgesandt wurde. Verschiedene Großmächte
	der Erde haben andere Herangehensweisen. Im Fall unserer Geschichte hätte eine erfolgreiche Mitteilung
	der Raumschiffscrew an die Erde mit den gesammelten Informationen über Go’Kra, dem Planeten der Chia,
	vermutlich einen Krieg um Lebensraum zwischen Erde und Go’Kra nach sich gezogen.
	
	
	\subsection{Motor}
	Der Motor besteht nun also aus der Möglichkeit, innerhalb dieser Welt unendlich viele Geschichten erzählen zu können, von denen jede von einem anderen Forschungsraumschiff und seinen Abenteuern in einer fremden Welt handelt.
	
	
	\subsection{Allegorie}\label{All.}
	Es finden sich mehrere Allegorien in dieser Welt. Die wohl allgemeinste ist das Star Trek nachempfundene Bild des Menschenraumschiffs, das die Weiten des Weltalls erkundet und auf die wundersamsten Planeten trifft.\\
	\\
	Eine weitere Allegorie ist die Kolonialisierung, ausgeweitet von Kontinenten auf der Erde zu Planeten in der Galaxie. Je nachdem, zu welcher Fraktion auf der Erde das jeweilige Forschungsschiff, von dem erzählt wird,
	gehört, finden sich andere Herangehensweisen an den Umgang mit den auf dem Zielplaneten bereits ansässigen
	Aliens. Interessant ist hierbei auch, dass die Kolonialisten zuerst einmal im Nachteil sind, da die Alienvölker
	immer hochentwickelt sind, sonst hätten sie es nicht dorthin geschafft.\\
	\\
	Ein Motiv, das sich ebenfalls konstant durch das Spiel zieht, ist die Hoffnung. Sie wird ein ums andere Mal aufgebaut, erdrückt, wieder aufgebaut, wieder erschlagen und wieder aufgebaut.\\
	Solange es nur ein kleinstes Bisschen Hoffnung zu geben scheint, klammern wir uns an diese, so gering die Chancen auch sein mögen. Diese Hoffnung treibt den Spieler an, das Spiel weiterzuspielen.\\
	Die Figur Bernd ist ein Beispiel dafür, was geschieht, wenn man die Hoffnung aufgibt.\\
	\newpage
	\section{Story}\label{story}
	\subsection{Erster Akt}
	Der erste Akt unserer Story ist auch die Vorgeschichte und erklärt wie es zu den
	Ereignissen in unserem Spiel kommt und wie die Hauptfigur beteiligt wurde.\newline
	Unsere Hauptfigur Jordan Collins ist ein von den United Western Nations, einem Zusammenschluss der europäischen Union und den Vereinigten Staaten, angestellter Astrophysiker. Seine Hauptaufgabe ist die Entdeckung neuer Planeten mit für Menschen lebensfähigen Bedingungen. Zufällig bekommt er ein geheimes Gespräch einer Gruppe von hochrangigen Vorgesetzten und Geoforschern mit, in dem es um den Beweis geht, dass die Erde in absehbarer Zukunft durch den Klimawandel nicht mehr lebensfreundlich ist. Die Gruppe bemerkt, dass Jordan das Gespräch mitangehört hat und meldet dies dem Sicherheitsdienst der Institution, von welchem Collins dann festgenommen wird. Da es im Interesse des Staates ist diese Erkenntnis vor der Bevölkerung geheim zu halten wird Collins mit der Sicherheit seiner Familie erpresst einem Projekt beizutreten, das dazu dient einen neuen Lebensraum für die Menschheit zu finden.\newline
	Diese Unternehmung, unter dem Alias Projekt Aphrodite, wurde vom Staat ins Leben gerufen, da man auf der Venus Beweise gefunden hat die darauf hindeuten, dass dort früher Leben existierte. Dieses wurde aufgrund der Klimaerwärmung gezwungen, einen neuen Lebensraum zu finden. Man ist sich bewusst, dass die Erde irgendwann dasselbe Schicksal ereilt, weshalb die Aufgabe des Projekts darin besteht mit einem Raumschiff und einer Crew über einem Zeitraum von mehreren Jahrzehnten, in denen sich die Crew im Cryoschlaf befindet, zu einem Planeten eines anderen Sonnensystems zu fliegen um dort zu untersuchen, ob auf diesem Planeten lebensfähige Bedingungen herrschen.\newline
	Jordan Collins steht jetzt vor dem Problem wie er sich entscheidet, da er entweder seine Familie in Gefahr bringt oder seine Familie alleine zurücklässt und sie eventuell nicht wieder sehen wird, entscheidet sich aber mitzumachen um für seine Familie und die Menschheit vielleicht ein Überleben auf einem neuen Planeten zu ermöglichen und er wegen seiner Ausbildung als Astrophysiker bereits gute Voraussetzungen für diesen Job mitbringt.\newline
	Das zufällige Mithören des Gesprächs ist im 3-Akt-Schema der Auslöser unserer gesamten Story mit der Frage, ob er mitmachen und seine Familie zurücklassen soll. Die Entscheidung mitzumachen ist dann der erste Wendepunkt, da er danach nicht mehr zurück kann.\newline 
	Um sich auf einen eventuell feindseligen und schwer bewohnbaren Lebensraum vorzubereiten, ist Jordan teil eines Vorbereitungsprogramm in dem er kämpferische und überlebensnotwendige Fähigkeiten erlernt.\newline 
	Nachdem dieses Training abgeschlossen war beginnt die Mission damit, dass sich die Crew in die Cryokapseln begibt und das Raumschiff fliegt mit Autopilot zum neuen Sonnensystem. Aufgrund eines Fehlers des Raumschiffs wacht die Crew nicht rechtzeitig auf und das Raumschiff stürzt auf dem Ziel-Planeten, Aphelios, ab, wodurch die meisten Crewmitglieder sterben. Auf diesem Aphelios existiert bereits Leben und die Aliens transportieren die Cryokapseln in ein Labor um die für sie neue Spezies zu untersuchen.\newline
	Diesen ersten Akt erlebt der Spieler aber nicht zu Beginn des Spiels, sondern man startet mit dem zweiten Akt und erfährt dann durch Rückblicke, Objekte, welche man im Spiel findet, sowie durch die Gedanken der Hauptfigur, wieso man in der Situation ist und wie die Welt um einen herum aussieht.\newline
	\subsection{Zweiter Akt}
	Zu Beginn des Spiels und auch des zweiten Akts liegt Jordan wie er auf einem Labortisch, von Chia umringt. Bei diesen Experimenten erlebt Jordan die Ereignisse des ersten Aktes in Form seiner Erinnerungen.\newline
	Die erste Hälfte des zweiten Aktes erzählt von der Flucht aus dem Labor. Dabei findet er den Großteil der Crew in Kapseln gelagert tot auf. Jordan läuft einem anderen Überlebenden namens Bernd über den weg, welcher Maßgeblich an seiner Erpressung beteiligt war. Bei einem Gespräch stellt sich heraus, das Bernd wahnsinnig geworden ist was dazu führt das Jordan nach einer Waffe greift und Bernd erschießt. Auf seinem weiteren Weg nach draußen wird ihm mehrmals geholfen, jedoch erfährt weder Jordan noch der Spieler wer oder was ihm hilft.\newline
	In der zweiten Hälfte irrt Jordan in der Stadt der Chia herum, auf der Suche nach seinem Raumschiff. Dabei gelangt er in eine Kirche, wo ihn ein Mitglied der Pandora erwartet. Bei einem Gespräch mit diesem erfährt er von der Religion der Chia, das dieses ihm aus dem Labor geholfen hat und die Pandora seine Unterstürzung wollen. Jordan lehnt das Angebot ab, da er seiner Pflicht die Erde zu kontaktieren nachkommen will, in der Hoffnung das er so überlebt.\newline
	\subsection{Dritter Akt}
	Im dritten Akt erreicht Collins schlussendlich das Raumschiff, in welchem er am Bordcomputer die eingegangenen Nachrichten der Erde durchliest. Durch die Nachrichten der UGN, einer aus der UWN entstandenen globalen Vereinigung aller Nationen der Erde, erfährt er, dass die Situation auf der Erde immer schlimmer wird und die Erde früher als erwartet nicht mehr bewohnbar sein wird. In der letzten Nachricht der UGN steht, dass der Planet zum Zeitpunkt seiner Ankunft kein lebensfähiger Planet mehr ist und die letzten überlebenden Menschen ebenfalls auf anderen Raumschiffen sind, die zu anderen Planeten geschickt wurden.\newline
	Mit dieser Nachricht wird Jordan klar, dass er zwar sein noch funktionsfähiges Raumschiff erreicht hat, es ihm aber nichts nützt, da es keinen Ort gibt, an den er zurückkehren könnte und er alleine auf einem fremden Planeten ist, ohne Wissen wo die restlichen Menschen hin geflogen sind.\newline
	Durch seine persönlichen Nachrichten erfährt er, dass seine Familie es auf eines der Raumschiffe geschafft hat und von ihrem Reiseziel, welches 25 Lichtjahre von Aphelios entfernt ist.\newline
	An diesem Punkt spaltet sich die Geschichte in drei Enden auf, aus denen der Spieler zwei wählen kann:
	\begin{enumerate}
		\item Jordan entscheidet sich dazu, den Planeten zu verlassen und zu seiner Familie zu fliegen, da er glaubt ungefähr zeitgleich mit ihnen anzukommen. Unglücklicherweise kamen sie früher an und entscheiden sich ihrerseits nach Aphelios aufzubrechen. Damit verpassen sie sich. Als Jordans Familie auf Aphelios landet, werden sie von den Chia überfallen und getötet.
		\item Jordan entscheidet sich dazu, sich den Pandora anzuschließen und findet in ihnen nach und nach neue Freunde. Nach einigen Jahren landet überraschenderweise ein Raumschiff auf Aphelios mit weiteren Menschen. Darunter ist auch Jordans Familie. Zusammen schaffen es die Pandora mit den Neuankömmlingen das System zu stürzen und einen neuen Lebensraum für Chia und Mensch zu schaffen.
	\end{enumerate}
	Der Besuch der Kirche ist nur optional. Sollte der Spieler daher nicht von den Rebellen erfahren haben wird das zweite Ende mit dem Folgenden ersetzt:
	\begin{enumerate}
		\setcounter{enumi}{2}
		\item Jordan entscheidet sich dazu, auf Aphelios zu bleiben. Er versucht sich auf der feindlichen Welt zurechtzufinden, scheitert aber letzten Endes da er in seinem Unterschlupf von einer Wache entdeckt und erschossen wird.
	\end{enumerate}
	\newpage
	\section{Derivat}\label{derivat}
 		Im folgenden Abschnitt wollen wir uns mit dem Derivat beschäftigen. Insbesondere wird die erzählte Geschichte zusammengefasst und die Stellung zur Main-Story, sowie der erhoffte Effekt des Derivats genauer erläutert.
 	\subsection{Story}
 		Zuerst ein mal möchten wir die Story, die wir im Derivat erzählen wollen, zusammenfassen um eine Basis für den nächsten Teil zu schaffen. Unsere Geschichte handelt von einem Kolonialist der Chia Namens Cho'Brok. Dieser ist der erste, der die neue Lebensweise unserer Spezies entdeckt und die anderen Mitglieder der Kolonie zu dieser konvertiert. Die Erzählung beginnt mit einer Beschreibung der alten Heimwelt der Chia, die Venus. Es wird die Ausgangssituation beschrieben, in der sich die Bevölkerung der Venus, viele Jahre vor Beginn der Menschheit, befindet. Die Erzählung fährt mit der Ankunft auf der neuen Heimwelt, genannt \glqq Go'Kra\grqq. Cho'Brok ist unzufrieden damit wie die Chia sich auf dem neuen Planeten verhält und beginnt eine Wanderung um nach Antworten zu suchen. Ohne eine Idee was er sucht begibt er sich, nach einer längeren Erkundung des Planeten, in eine Höhle. Hier sieht Cho'Brok drei Visionen. Die Erste zeigt wie die Chia zusammen die Erlösung erfahren. In der zweiten Vision sind die Chia ausgestorben was dazu führt das sie die Erlösung nicht erreichen. In der letzten Vision sieht man wie fremde Wesen auf dem Planeten landen und die Chia von ihrem Weg zur Erlösung abbringen. Seine Gedanken schreibt er auf die Wände der Höhle. Cho'Brok geht zurück in die Kolonie und predigt den restlichen Chia von seinem Erlebnis. Diese beginnen nach und nach seinen Reden zu folgen wodurch sich die Religionsgemeinschaft bildet. Daraus entsteht die zum Zeitpunkt des Spiels vorherrschende Regierung.

 	\subsection{Form}
 		Bei der gewählten Form des Derivats handelt es sich um eine bibel-ähnliche und märchenhafte Erzählung. Das Vorbild für diese ist die Geschichte des brennenden Dornenbusch, welche, laut der Bibel, besagt das Gott Moses den Auftrag das Volk Israel aus Ägypten in das heilige Land zu führen.
 	\subsection{Erhoffte Effekte}
 		Das Derivat soll primär drei Effekte erzielen, die im folgenden Abschnitt erläutert werden:
 	\begin{enumerate}
	\item{\textbf{User-Bindung}}\newline
	 	Durch die Erzählung woraus die Hintergründe der Chia-Religion hervorgehoben werden, soll der User eine besseres Verständnis für die Handlung der Chia gegenüber dem Spieler, sowie untereinander und zur Regierung haben. Die Gründe warum sie so xenophob gegenüber unseres Forschers eingestellt sind, sowie ihre Vergangenheit und die Beweggründe der anderen Chia-Parteien werden hoffentlich deutlicher. Dies kann zwar die gewünschte Horror-Stimmung etwas entschärfen, sorgt aber für einen glaubwürdigeren Gegner, welcher aus eigenem Sturm und Drang handelt und nicht nur ein stumpfes Monster ist was nur töten möchte, wie zum Beispiel der Predator oder der Xenomorph aus den Alien-Filmen.
	 \item{\textbf{Vermarktung}}\newline
	 	Die Erzählung richtet sich an diejenigen User, die mehr Informationen zu den Chia möchten und sich, wie bereits im oberen Abschnitt erwähnt, mehr für die Hintergründe der Religion interessieren. Es bestünde die Möglichkeit mehrere solche Texte anzufertigen, die als Basis für eine eventuelle Fan-Religion dienen könnte.
	 	\item{\textbf{Einfluss auf die Gesamterzählung}}\newline
	 	Auf die Gesamterzählung wird der Effekt nicht wirklich groß ausfallen. Die Hintergründe für die Religion der Chia haben keinen besonders großen Einfluss auf das Gameplay, ebenso wie die Gesamterzählung des Spiels. Sie soll eher das Gesamte erweitern und sinnvoll ergänzen. Der User soll keine eingeschränkte Spielerfahrung haben, nur weil er das Derivat nicht kennt, wohingegen derjenige, der sich mit diesem befasst, etwas mehr Verständnis für die Chia hat. Im Spiel selbst finden sich Elemente aus dem Derivat wieder, welche für den Spieler einen Wiedererkennungswert haben.     
 	\end{enumerate}
 	\subsection{Main-Story gegen Derivat}
	 Im letzten Teil dieses Abschnitts möchten wir darstellen, wie das Verhältnis zwischen der Story, des Derivats und der eigentlichen Hauptstory des Spiels steht. Wie bereits angesprochen steht die Derivatsgeschichte unabhängig von der Hauptstory und soll das Spielerlebnis wenn nur schwach beeinflussen. Sie soll viel mehr dafür sorgen, dass das von uns gebaute Universum, sowie die Chia besser zu verstehen sind. Zeitlich sind die Hauptstory und die Erzählung des Derivats sehr weit auseinander. Es liegen viele Jahrhunderte und Generationen von sogenannten heiligen Gesandten dazwischen, wodurch die tatsächlichen Ereignisse näher an einem Mythos als an einer historisch korrekten Erzählung liegen, was die Auslegung der Geschichte besser an die Regierung anpasst.  	
			
\end{document}
